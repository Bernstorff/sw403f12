\chapter{Parser}

Something introduction

Requirements:
\begin{itemize}
	\item The parser should be able to receive tokens, which can then be converted into parse trees, 
	\item Report useful error messages, if the user has written something illegal according to the EBNF (Extended Backus�Naur Form). 
	\item The parser should be LL(1) and not be ambiguous, to not make it find more than one parse tree. 
\end{itemize}

The following is the EBNF of the language:

\begin{spverbatim}
SS        	: {Blocks} 
Blocks    	-> BeginBlock {Lines} EndBlock 
          	| SettingBlock
Lines    	-> SettingBlock 
              | Numeration
              | Itemlist 
          	| Plains eol
Numeration   -> nlist (Plains eol | Numeration)
Itemlist 	-> blist (Plains eol  | Itemlist)
BeginBlock   -> beginkwd BEBlock eol
EndBlock 	-> endkwd BEBlock Eols
BEBlock  	-> lcurly (Ident Spaces | EPSILON) pipe Ident Spaces rcurly
SettingBlock -> settingkwd lcurly ShortIdent Spaces pipe String rcurly eol
Plains   	-> (ShortBlock | CharSpace | Number)$^+$
ShortBlock   -> Kwd lcurly {ShortIdent} pipe Plains rcurly
ShortIdent   -> Kwd colon (Ident | Number) Spaces
String   	-> CharSpace$^+$
CharSpace	-> char 
          	| space
Ident    	-> char
Number   	-> digit
Kwd      	-> atsign Ident
Spaces   	-> space
              | EPSILON
Eols     	-> eol+
\end{spverbatim}

This grammar is able to construct everything that is required of the language. \\
Let's look at one of the previous examples:

\begin{verbatim}
Input:
@begin{fade|slide}
    Hello World
@end{|slide}
\end{verbatim}

The parse tree for this would look like: 

\begin{figure}[! h]
\centering
	 \includegraphics[width=300px]{images/Parsetree.png}
		 \caption{Parsetree for a simple slide. Using the program kfG}	
	\label{fig:Parsetree}
\end{figure}

The figure \ref{fig:Parsetree} is generated by the program ``kfG��, which has a slightly different structure, because each ``plain�� does not contain multiple characters. The figure shows how it would parse the tree, with the input in a slide of �Hello World''. The kfG structure can be seen in appendix \ref{AkfG}. We have chosen not to show anything that is more complex than the parse tree\ref{fig:Parsetree}, because the complexity would increase greatly.