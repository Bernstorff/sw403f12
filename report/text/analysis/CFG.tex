\chapter{CFG}
% ``''	<-- how to make quotes
Syntax:

The following example are some of the basic elements in the language:
Input:

\begin{verbatim}
@begin{fade|slide}
    Hello World
@end{|slide}
\end{verbatim}

Output: Hello World
\\ \\
Which says that a slide begins, with a transition ``fade'', and that the slide contains the text ``Hello World''. \\
After an @, a keyword always begins, making it easy for the compiler to recognize a keyword.
\\ \\
Here is another keyword ``Setting'', in this example the setting is set for the type ``title'' to a font size of 40 points. \\
A setting can be initiated outside, as well as inside, a slide. The difference is that, outside a slide the setting is applied to all the upcoming slides, if the setting is inside the slide, it is only applied to that slide.

\begin{verbatim}
Input: @setting{@font_size:40|title}
\end{verbatim}

Output: A way to use the title type can be by the following example:

\begin{verbatim}
Input:
@begin{fade|slide}
    @title{|Hello and welcome}
@end{|slide}
\end{verbatim}Output: Hello and welcome (in the title format)

As you see, there are nothing on the left side of the pipe (The sign before ``Hello''). Between the left curly and the pipe, settings can be applied to that specific sentence, like:

\begin{verbatim}
Input: @title{@font_size: 70|Hello and welcome}
\end{verbatim}

Output: Hello and welcome (in title format with size 70)
\\ \\
Which sets ony this line of type ``title'' to font size 70, instead of 40 which was initialized above. This can also be applied to normal sentences like:

\begin{verbatim}
Input: @apply{@font_size:25 | Welcome to this slide}
\end{verbatim}

Output: Welcome to this slide (with fontsize 25)
\\ \\
You have to use the keyword ``apply'', to change the format of a regular text. Here the font size is set to 25, and the text is ``Welcome to this slide''.
\\ \\
The weight of text can be done in two ways, the first looks like the way we just changed the font size:

\begin{verbatim}
Input: @apply{@font_weight:bold | Welcome to bold text}
\end{verbatim}

Output: \textbf{Welcome to bold text}
\\ \\
A quicker way to set a bold text is as follows:

\begin{verbatim}
Input: @b{|Welcome to bold text}
\end{verbatim}


Output: \textbf{Welcome to bold text}
\\ \\
Furthermore, we now want to underline a single word in that sentence:

\begin{verbatim}
Input: @b{|Welcome to @u{|bold} text}
\end{verbatim}

Output: \textbf{Welcome to \textit{bold} text}
\\ \\
Which make the whole text bold and underlines the word ``bold''.
\\ \\
Combining them, would look like this:

\begin{verbatim}
Input: @apply{@font_weight:bold | Welcome to @u{|bold} text}
\end{verbatim}

Output: \textbf{Welcome to \textit{bold} text}
\\ \\
And gives the same result as the last one (eg. ref).
\\ \\
The following a larger example to demonstrate what is just used:

\begin{verbatim}
Input:
@begin{fade|slide}
    @title{|@b{Welcome to this course}
    @Setting{@font_type: Arial | text}
    This course will contain information about how you @u{underline} things, and how you do other    
    @i{|weight stuff} on sentences.
    @apply{@font_weight:bold | Like this}
@end{|slide}
\end{verbatim}
Output:
Welcome to this course (as title)
This course will contain information about how you \underline{underline} things, and how you do other \textit{weight stuff} on sentences. (with font type Arial)
\textbf{Like this.}  (with font type Arial)

A setting does not have to be set at all, because if it is not set, the standard setting will be used

\subsection{Lists}
There are two types of lists, a list consists of either bullets or numerations. \\
A bulletlist is made by writing the following:

\begin{verbatim}
Input:
List of things to buy
*  Milk
*  Wolemeal bread
** Light bread
*  Coffee
\end{verbatim}

Output:
\begin{itemize}
\item Milk
\item Wolemeal bread

\begin{itemize}
\item Light bread
\end{itemize}

\item Coffee
\end{itemize}

Where a star symbolises a bullet, and two bullets symbolises a bullet inside a bullet.
Numeration is made by the following:

\begin{verbatim}
Input:
Agenda
#  Introduction
#  Presentation
## Code examples
#  Evaluation
\end{verbatim}

Output:
\begin{enumerate}
\item Introduction
\item Presentation
\begin{enumerate}
\item Code examples
\end{enumerate}
\item Evaluation
\end{enumerate}

Where a \# symbolises a number that incrementing. By using two \# creates a sub numeration starting from one.