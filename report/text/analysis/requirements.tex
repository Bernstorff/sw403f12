\section{Requirements}
The following requirements are set for the language to consider it as a complete language:
\\ \\
\noindent{\textbf{Capabilities}} \\
\begin{itemize}
\item The language has to make it possible to make bullet points and enumerations.
\item The language has to make it possible to import pictures from the Internet.
\item The user should be able to change font- type, color, size and line height.
\item The user should be able to make a transition between each slide. 
\end{itemize}
\textbf{Error handling}\\
\begin{itemize}
\item The language should be able to tell the user which line an error has occurred. 
\end{itemize}
\textbf{Test}\\
\begin{itemize}
\item The language should be easy to write in, determined by the test:
\begin{itemize}
\item The experienced user of the language should be able to make five slides, with decent content, using standard settings, within 10 minutes, where the content is prewritten.
\end{itemize}
\end{itemize}

\subsection*{Limitation}
It was estimated that it could become a source of distraction if people had too many options in the language, which would lead to a decrease in slides made per minute.

\begin{itemize}
	\item We have decided not to implement tables in our language due to the complexity of this feature.
	\item Audio is another feature, which we are not implementing in the language, because of its complexity, and because audio is not a key factor in our slide show language.
% \item Transition of text in the slidewhoe is not necessary for a slide show, which is why it will be left out, in the first place.
	\item Animation of text in the slideshow is not necessary for a slide show, which is why it will be left out, in the first place.
\end{itemize}