\chapter{Problem formulation}
The task of making a slide show in applications like Microsoft�s Power Point or Apple�s Keynote is a mouse based task. How can you make a better alternative to Latex Beamer? How can you create a programming language in which a user can make a slide show presentation without the need for a pointing device, and not have to think about the layout of single slides, but only define the general layout. Furthermore, how can you display the presentation in a way so that it will look equal on all computers. \\
Challenges include creating a suitable domain specific language with weight on consistency and simplicity, enabling the user to focus on the content rather than the layout.
\\ \\
The following questions provide a limitation for the scope of the project:
\\ \\
Which platform should be used for presentation? \\
Who are the users of the language? \\
Which layout decisions does the user need? \\
What are the language limitations? \\
What can be expressed in the language?
\\ \\
\section{Known slideshow application}
In this section some of the best known applications, Microsoft PowerPoint, Apple Keynote and Latex Beamer, is compared to the solution developed during this project.
\\ \\
Microsoft PowerPoint and Apple Keynote are slideshows application which uses drag-and-drop functions, why they will not be focused more on, because this approach to create slideshows violates the requirements of being a non-pointing device based application, specified in the problem statement of this report. Because these violate the problem statement, the main focus will be on the differences between Latex Beamer and the slideshow programming language developed in this project.
\\ \\
\section{Beamer}
An example of Latex Beamer is as follows:

\begin{lstlisting}[frame=single, caption={Beamer example}, label=lst_beamer]

\textbf{Main document:}
\begin{document}

\include{Slide_document.tex}

\end{document}


\textbf{Slide_document:}
\frame {
	\frametitle{Welcome to this course}

\textit{\fontfamily{uarial}\selectfont {This course will contain information about how you \underline{underline} things, and how you do other \textit{weight stuff} on sentences.} \\
\textbf{\fontfamily{uarial}\selectfont {Like this}} \\	}
}

%Output:%
%\textbf{Welcome to this course} (as title)%
%This course will contain information about how you \underline{underline} things, and how you do other \textit{weight stuff} on sentences. (with font type Arial)%
%\textbf{Like this.}  (with font type Arial)%
\end{lstlisting}
The example seen in listing \ref{lst_beamer}, is the Beamer-code for expressing the output. The Beamer-code seen here is only to express output. To make a slideshow using Beamer you have to set up a main document. In this document, all the settings about what theme, colours, inputs (other files), etc., for the slideshow is set up. The editor (TexMaker) only generates a very small amount of the main document, which leaves a lot of setup for the user, if additional settings is wanted. If only a general slideshow is required, the main document will not need much work. A general slideshow is without colours, themes or the need for additional packages.\\

Compared to the Beamer-code the developed should be made more compact to make slides faster to express.
\\ \\
A test between Latex Beamer and the developed language will be made to determine which and why each language is better than the other.