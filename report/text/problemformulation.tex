\section{Initiating problem:}
How can it be made easier to prototype card games on a computer? 
\chapter{Problem formulation}
We want to create a domain specific language for card games to enable rapid prototyping. With a 
general programming language, trivial tasks would require much effort with little gain, since card 
representation and rule expressions would have to be written from the ground up. The language should be targeted at programmers with experience in imperative and object oriented programming. With a domain specific language we create an abstraction for these trivial tasks and enable the programmer to focus on the rules and behavior of the game, making prototyping easier and shortening the development time. 
Challenges include creating a suitable abstraction without limiting the number of games expressible
by the language, or creating an almost general-purpose language. The language should be easy to write and debug, since the focus is rapid prototyping.

The following questions provide a limitation for the scope of the project:

\begin{itemize}
\item How can we make a programming language for rapid prototyping card games ?
\item What does a programming language require to be domain specific ?
\item How should a domain specific language be structured ?
\end{itemize}



Initiating problem
How can a language remove the need for a pointing device when creating slide shows?

Problem formulation

The task of making a slide show in applications like Microsoft’s Power Point or Apple’s Keynote is a mouse based task. We want to create a programming language wherein a user can make a slide show presentation which is purely text based, and not have to think about the layout of single slides but only define the general layout. Furthermore, we want to display the presentation in a way so that it will look equal on all computers. 
Challenges include creating a suitable domain specific language with weight on consistency and simplicity, enabling the user to focus on the content rather than the layout.

The following questions provide a limitation for the scope of the project:

Which platform should be used for presentation?
Who is the users of the language?
Which layout decisions does the user need ?
What are the language limitations ? What can be expressed in the language ?