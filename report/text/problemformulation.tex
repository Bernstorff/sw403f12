\section{Problem formulation}
Before designing the language, an analysis of the user should be conducted to determine the language requirements best suited to the user. Depending on the user a certain type of card type can be defined, to decrease the complexity of the code. To illustrate the complexity of the language, a diagram should be included (the learning curve) to give an impression of how difficult the language is to learn compared to time used. Furthermore, the priorities of the language should be established to make the best language for the user, with the purpose the language is designed under.

The following questions is to form a focus point and should provide as a limitation for the project. These questions will be answered throughout the report. 

\begin{itemize}
	\item Who is the main users of this language, and why?
	\item What level of verbosity is needed to make the language easily read- and debug-able?
	\item What is the priorities, focus and the limitation of the language?
	\item What is the assumed learning curve of the language, and why?
\end{itemize}